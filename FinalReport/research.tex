\section{Existing Systems}
\subsection{Alchemy API}

\section{WordNet}
A synset is a set of synonyms. 
WordNet can be interpreted and used as a lexical ontology [Wikipedia]. However such an ontology should be corrected before use since it contains hundreds of basic semantic inconsistencies such as 
i) the existence of common specializations for exclusive catgories and 
ii) redundancies in the specialization hierarchy.
Furthermore, transformation of WordNet to an ontologyshould involve 
i) distinguishing the specialization relations subtypeOf and instanceOf relations, an
ii) associating intuitive unique identifiers to each category

Main relation among words in WordNet is synonym (as between the words shut and close, or car and automobile). These get grouped into unordered set - synsets. WordNet has 117000 synsets

A hyperonym/hypernym is the super-subordinate relation. It links a word such as furniture, to bed like this \{furniture, peice\_of\_furniture\}. This is transitive. If Armchair is a kind of chair, and if a chair is a kind of furniture, then armchair is a kind of furniture. Instances are always leaf nodes in their hierarchies. 

Meronymy, the part-whole relation holds between sysets like {chair} and {back, backrest}, \{seat\} and \{leg\}. Parts are inherited from their superordinates. If a chair has legs, an armchair has legs. They don't go upwards. E.g chair and counter are both furniture, a chair has legs, but furniture doesn't inherit taht upwards because that would mean counter has to inherit it and it doesn't.

Verb synsets are arranged into hierarchies as well. Verbs towards the bottom of the trees express increasingly specific manners characterizing an event, as in {communicate}-{talk}-{whisper}. The specific manner expressed depends on teh semantic field it's in, in this case, volume. Others are speed: {move}-{jog}-{run}, or intensity of emotion: {like-love-idiolize}.

Part of speech (POS). Wordnet cosists of four sub-nets, one each for nouns, verbs, adjectives and adverbs, with few cross-POS pointers. Cross-POS relations include the morphosemantic links that hold among semantically similar words sharing a stem with the same maeaning: observe(verb), observant(adj) observation, observatory (nouns). 


\subsection{Classes}
Word: 
getSynset

\section{Stanford Parser}
%Go through the manual and find where it points out a flaw. Such as not developed thoroughly

\section{Sentiment Analysers}
Most sentiment analysers available are aimed at picking at a speicific topic and assessing the feelings towards it. 
%My project aims to find categories in teh text. Later assessing who enjoys it and whetehr they get a 1 or a 0 for the category

\section{Literature Review}
\subsection{Recognizing Contextual Polarity in Phrase-Level Sentiment Analysis}
%Basically verbatim from paper. Rewrite. 
This paper looks at how phrase level sentiment analysis can determine positive and negative expressions~\cite{phraselevelsentimentanalysis}. 
We can pick out keywords in a phrase and look at their {\bf prior polarity}. This is their polarity before we put it in context. This paper discusses new experiments in automatically distinguishing prior and contextual polarity: They use a two step process that uses machine learning and a variety of features. 
\begin{itemize}
\item First step classifies each phrase containing a clue as neutral or polar.
\item Second step takes all phrases marked in step one as polar and disambiguates their contextual polarity.
\end{itemize}
The system can now automatically identify contextual polarity for a large subset of sentiment expressions.
\subsection{Combining strengths, emotions and polarities for boosting Twitter sentiment anlaysis}
This paper proposes an approach for boosting Twitter sentiment classification using different sentiment dimensions as meta-level features~\cite{combiningstrengthsemotionspolarities}. They combine aspects such as opinion strength, emotion and polarity indicators, generated by existing sentiment analysis methods and resources. The combination provides significant improvement in Twitter sentiment classification tasks such as polarity and subjectivity. 

Emoticons can introduce noise~\cite{combiningstrengthsemotionspolarities}.

\section{Tools}
\subsection{Stanford Natural Language Parser}
The parser tokenises the text and then tags each token with what it means gramatically i.e a verb, noun, adverb etc. Builds a parse tree for it. 